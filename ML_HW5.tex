\documentclass[12pt]{article}

\usepackage{url,graphicx,tabularx,array,geometry}

\setlength{\parskip}{1ex} %--skip lines between paragraphs
te LaTeX source file for homework problem solutions.
% Alan T. Sherman (9/9/98)

% Running LaTeX
%
% Name this file FOO.tex
% latex FOO
% latex FOO   
%    (You have to run latex twice to get the cross references correct.
%     Running latex creates a file FOO.dvi 
%     You can view dvi files with the program xdvi )
% xdvi FOO.dvi &
%
% lpr -d FOO.dvi
%    (To print the dvi file.   Be sure to use the "-d" print option,
%     and be sure your printer can handle dvi files (not all printers can).
%     Do NOT print with "lpr FOO.dvi", which will print tens of pages
%     of unreadable dvi source code. Printing a postscript (ps) file
%     is usually more reliable, as explained below.)
%
% dvips FOO.dvi
%    (To create a postscript file named FOO.ps 
%     which you can view with the program ghostview )
% ghostview FOO.ps &
% lpr FOO.ps
%    (To print the ps file.)

%%%%%%%%%%%%%%%%%%%%%%%%%%%%%%%%%%%%%%%%%%%%%%%%%%%%%%%%%%%%%%%%%%%%%%

\documentstyle[12pt]{article}

% Set the margins
%
\setlength{\textheight}{8.5in}
\setlength{\headheight}{.25in}
\setlength{\headsep}{.25in}
\setlength{\topmargin}{0in}
\setlength{\textwidth}{6.5in}
\setlength{\oddsidemargin}{0in}
\setlength{\evensidemargin}{0in}

%%%%%%%%%%%%%%%%%%%%%%%%%%%%%%%%%%%%%%%%%%%%%%%%%%%%%%%%%%%%%%%%%%%%%%%
% Macros

% Math Macros.  It would be better to use the AMS LaTeX package,
% including the Bbb fonts, but I'm showing how to get by with the most
% primitive version of LaTeX.  I follow the naming convention to begin
% user-defined macro and variable names with the prefix "my" to make it
% easier to distiguish user-defined macros from LaTeX commands.
%
\newcommand{\myN}{\hbox{N\hspace*{-.9em}I\hspace*{.4em}}}
\newcommand{\myZ}{\hbox{Z}^+}
\newcommand{\myR}{\hbox{R}}

\newcommand{\myfunction}[3]
{${#1} : {#2} \rightarrow {#3}$ }

\newcommand{\myzrfunction}[1]
{\myfunction{#1}{{\myZ}}{{\myR}}}


% Formating Macros
%

\newcommand{\myheader}[4]
{\vspace*{-0.5in}
\noindent
{#1} \hfill {#3}

\noindent
{#2} \hfill {#4}

\noindent
\rule[8pt]{\textwidth}{1pt}

\vspace{1ex} 
}  % end \myheader 

\newcommand{\myalgsheader}[0]
{\myheader{UMBC, Department of Computer Science and Electrical Engineering}
{CMSC-441 Algorithms}{Fall 1998}{Section 0101}}

% Running head (goes at top of each page, beginning with page 2.
% Must precede by \pagestyle{myheadings}.
\newcommand{\myrunninghead}[2]
{\markright{{\it {#1}, {#2}}}}

\newcommand{\myrunningalgshead}[2]
{\myrunninghead{CMSC-441 Algorithms}{{#1}}}

\newcommand{\myrunninghwhead}[2]
{\myrunningalgshead{Solution to HW {#1}, Problem {#2}}}

\newcommand{\mytitle}[1]
{\begin{center}
{\large {\bf {#1}}}
\end{center}}

\newcommand{\myhwtitle}[3]
{\begin{center}
{\large {\bf Solution to HW {#1}, Problem {#2}}}\\
\medskip 
{\it {#3}} % Name goes here
\end{center}}

\newcommand{\mysection}[1]
{\noindent {\bf {#1}}}

%%%%%% Begin document with header and title %%%%%%%%%%%%%%%%%%%%%%%%%

\begin{document}

\myalgsheader

\pagestyle{plain}

\myhwtitle{1}{4 (3.2-2)}{Ben Bitdiddle}

\bigskip

%%%%% Begin Solution %%%%%%%%%%%%%%%%%%%%%%%%%%%%%%%%%%%%%%%%%%%%%%%%%%%

\noindent
We must find an asymptotic upper bound on the summation
\myzrfunction{S} defined by

% Here's an example of how to do display equations.
%
\begin{equation} \label{a}
S(n) = \sum_{k=0}^{\lfloor \lg n \rfloor} 
\left\lceil {{n} \over {2^k}} \right\rceil,
\end{equation}

\noindent whenever $n \in {\myZ}$.  

% First page break
%   Note: "myheadings" is a LaTeX keyword; it is not a user-defined entity.
%
\clearpage
\setcounter{page}{2}
\pagestyle{myheadings}
\myrunninghwhead{1}{4}

As a partial check on our work,  in Table~\ref{tvalues}
we computationally verify Equation~\ref{a}
for $n=1,2$.

% Here's an example of how to do tables.
%
\begin{table}[h] \label{tvalues}
\begin{center}
\begin{tabular}{|l|l|l|l|}
$n$ & $2n-2$ & $S(n)$ & $2n+\lg n + 1$ \\ \hline
$1$ & $2(1)-2=0$ &
$\lceil 1/2^0 \rceil = 1$ & $2(1) + \lg 1 + 1= 3 $\\
$2$ & $2(2)-2 = 2$ &
$\lceil 2/2^0\rceil + \lceil 2/2^1 \rceil = 2 + 1 = 3$ &
$2(2) + \lg 2 + 1= 6$ \\ \hline
\end{tabular}.
\end{center}

\caption{Values of $S(n)$ for $n = 1, 2$.}
\end{table}

\vfill
\noindent
{\it Acknowledgment:  I thank Mary Lou
for suggesting to use the method of splitting and bounding.}

%%% End solution and document %%%%%%%%%%%%%%%%%%%%%%%%%%%%%%%%%%%%

\end{document}
