\documentclass{beamer}

\usepackage[british]{babel}
\usepackage{graphicx,hyperref,ru,url}

% The title of the presentation:
%  - first a short version which is visible at the bottom of each slide;
%  - second the full title shown on the title slide;
\title[RU style for Beamer]{
  Linear-chain CRF for NLP in MADlib}

% The author(s) of the presentation:
%  - again first a short version to be displayed at the bottom;
%  - next the full list of authors, which may include contact information;
\author[Kun Li]{
  Kun Li \\\medskip
  {\small \url{kli@cise.ufl.edu}} \\ 
  {\small \url{http://www.cise.ufl.edu/~kli/}}}

% The institute:
%  - to start the name of the university as displayed on the top of each slide
%    this can be adjusted such that you can also create a Dutch version
%  - next the institute information as displayed on the title slide
\institute[University of Florida]{
  Department of Computer \& Information Science \& Engineering\\
  University of Florida}

% Add a date and possibly the name of the event to the slides
%  - again first a short version to be shown at the bottom of each slide
%  - second the full date and event name for the title slide
\date[slides Example 2010]{\today}

\begin{document}

\begin{frame}
  \titlepage
\end{frame}

\begin{frame}
  \frametitle{Outline}

  \tableofcontents
\end{frame}

% Section titles are shown in at the top of the slides with the current section 
% highlighted. Note that the number of sections determines the size of the top 
% bar, and hence the university name and logo. If you do not add any sections 
% they will not be visible.
\section{Introduction}

\begin{frame}
  \frametitle{Introduction}

  \begin{itemize}
    \item Conditional probability p(label sequence y|observation sequence x) rather than p(y,x)
    \item Allow arbitrary, non-independent features of the observation sequence X. 
    \item A CRF on (X,Y) is specified by a vector $f$ of local features and a corresponding weight vector
          lambda.
    \item Each local feature is either a $state feature s(y,x,i)$ or a transition feature t(y,y) 
  \end{itemize}
\end{frame}

\section{Background information}

\begin{frame}
  \frametitle{Background information}

  \begin{block}{Slides with \LaTeX}
    Beamer offers a lot of functions to create nice slides using \LaTeX.
  \end{block}

  \begin{block}{The basis}
    This style uses the following default styles:
    \begin{itemize}
      \item The CRF's global feature vector for input sequence x and label sequence y is given by
            $\mathbf{F}$(y,x)=
      \item whale
      \item rounded
      \item orchid
    \end{itemize}
  \end{block}
\end{frame}

\section{The important things}

\begin{frame}
  \frametitle{The important things}

  \begin{enumerate}
    \item This just shows the effect of the style
    \item It is not a Beamer tutorial
    \item Read the Beamer manual for more help
    \item Contact me only concerning the style file
  \end{enumerate}
\end{frame}

\section{Parallel Training CRFs in MADlib}

\begin{frame}
  \frametitle{Parallel Training CRFs in MADlib}
  \fbox{%
        \parbox{0.95\linewidth}{%
\textcolor{green}{\bf Parallel CRF training algorithm}\\
1.Evaluate the log-likelihood function and gradient vector\\
$\ell_{\lambda_k}=\lambda F(y_k,x_k)-\log Z_\lambda(x_k)$(log-likelihood for doc. $k$)\\
$\nabla \ell_{\lambda_k}=\lambda F(y_k,x_k)-E_{p\lambda(Y|x_k)}F(Y,x_k)$(gradient vector for doc. $k$)\\
2.Sum up over all documents \\
$\ell_\lambda=\sum_k\ell_{\lambda_k}-\frac{\lambda^2}{2\sigma ^2}+const$(log-likelihood for all docs.)\\
$\nabla \ell_\lambda=\sum_k\nabla \ell_{\lambda_k}-\frac{\lambda}{\sigma ^2}$(gradient vector for all docs.)\\
3.Perform LBFGS optimization, a variation of quasi-Newton algorithm\\
$lbfgs(\lambda,\ell_\lambda,\nabla \ell_\lambda,Hessian,other args)$\\
4.Repeat step 1 and step2 until stop condition is satisfied
        }%
}\\
\end{frame}

\section{Conclusion and Future Work}

\begin{frame}
  \frametitle{Conclusion and Future Work}

  \begin{itemize}
    \item SELECT madlib.traindata\_textfex(ags)
    \item SELECT madlib.linear\_crf\_training(ags)
  \end{itemize}
\end{frame}

\end{document}
