\documentclass[12pt]{article}

\usepackage[margin=1in]{geometry} 
\usepackage{amsmath,amsthm,amssymb}
\usepackage{enumerate}

\newcommand{\N}{\mathbb{N}}
\newcommand{\Z}{\mathbb{Z}}

\newenvironment{theorem}[2][Theorem]{\begin{trivlist}
\item[\hskip \labelsep {\bfseries #1}\hskip \labelsep {\bfseries #2.}]}{\end{trivlist}}
\newenvironment{lemma}[2][Lemma]{\begin{trivlist}
\item[\hskip \labelsep {\bfseries #1}\hskip \labelsep {\bfseries #2.}]}{\end{trivlist}}
\newenvironment{exercise}[2][Exercise]{\begin{trivlist}
\item[\hskip \labelsep {\bfseries #1}\hskip \labelsep {\bfseries #2.}]}{\end{trivlist}}
\newenvironment{problem}[2][Problem]{\begin{trivlist}
\item[\hskip \labelsep {\bfseries #1}\hskip \labelsep {\bfseries #2.}]}{\end{trivlist}}
\newenvironment{question}[2][Question]{\begin{trivlist}
\item[\hskip \labelsep {\bfseries #1}\hskip \labelsep {\bfseries #2.}]}{\end{trivlist}}
\newenvironment{corollary}[2][Corollary]{\begin{trivlist}
\item[\hskip \labelsep {\bfseries #1}\hskip \labelsep {\bfseries #2.}]}{\end{trivlist}}

\begin{document}

% --------------------------------------------------------------
%                         Start here
% --------------------------------------------------------------

\title{Database Management Systems(COP 5725) \\ HW1}%replace X with the appropriate number
\author{Instructor: Dr. Daisy Zhe Wang\\\\
TA:\\ 
Kun Li, Yang Chen\\ 
${kli,yang}@cise.ufl.edu$\\
} %if necessary, replace with your course title

\date{\today}
\maketitle


\begin {table}[h]
\[\begin{tabular}{|l|p{10cm}|} \hline
  Name: & \\ \hline
  UFID: & \\ \hline
  Email Adress: & \\ 
  \hline  
\end{tabular}\]
\end{table}

\linespread{2}
\noindent
Pledge(Must be signed according to UF Honor Code)\\
On my honor, I have neither given nor received unauthorized aid in doing this
assignment.\\

Signature \underline{\hspace{5cm}}

\null
\vfill
\begin{center}
\large \emph{Department of Computer and Information Science and Engineering}\\
\Large \emph{University of Florida}
\end{center}


\newpage
Exercise 1(ER)\\

The academic world is an interesting example of international cooperation and exchange.
This problem is concerned with modeling of a database that contains information on researchers, academic institutions, and collaborations among researchers. A researcher can
either be employed as a professor or a lab assistant. There are three kinds of professors:
Assistant, associate, and full professors. The following should be stored:

\begin{itemize}
\item
For each researcher, his/her name, year of birth, and current position (if any). 

\item 
For each institution, its name, country, and inauguration year.

\item 
For each institution, the names of its schools (e.g. School of Law, School of Business,
School of Computer Science,. . .). A school belongs to exactly one institution.

\item 
An employment history, including information on all employments (start and end
date, position, and what school)
\item
Information about co-authorships, i.e., which researchers have co-authered a research
paper. The titles of common research papers should also be stored.

\item
For each researcher, information on his/her highest degree (BSc, MSc or PhD), 
including who was the main supervisor, and at what school.

\item
For each professor, information on what research projects (title, start date, and end
date) he/she is involved in, and the total amount of grant money for which he/she
was the main applicant.
\end{itemize}

\begin{itemize}
\item
Draw an E/R diagram for the data set described above. Make sure to indicate all
cardinality constraints specified above. The E/R diagram should not contain redundant
entity sets, relationships, or attributes. Also, use relationships whenever appropriate. If
you need to make any assumptions, include them in your answer.
\item
Convert your E/R diagram from question a) into relations, and write SQL statements
to create the relations. You may make any reasonable choice of data types. Remember
to include any constraints that follow from the description of the data set or your E/R
diagram, including primary key and foreign key constraints.
\end{itemize}
 

\newpage
Exercise 2(Relational Algebra)\\
Suppose you got an internship next summer in Bay Area. And you want to book a plane ticket
from Gainesville to San Jose between date1 and date2. Consider the following schema for an airline database
(primary key attributes are in \textbf{bold})

\begin{itemize}
\item 
FLIGHTS($\textbf{flight\_num}$, \text{source\_city, destination\_city, distance})
\item
\text{DEPARTURES(\textbf{flight\_num, date}, plane\_type, price)}
\item 
\text{BOOKINGS(passenger\_id, flight\_num, date, seat\_number)}
\end{itemize}

Express the following queries in SQLs
\begin{itemize}
\item
Find the direct flights from Gainesville to San Jose.

\item 
Find the possible itinerary that that have one-stop flights from Gainesville to San Jose.

\item
Find the cheapest direct flight from Gainesville to San Jose.

\item 
Find the cheapest one-stop flights from Gainesville to San Jose.

\item
Find the passengers who booked a flight from Gainesville to San Jose.

\item
Find the flights with one stop which still are not full.

\end{itemize}

\newpage
Exercise 3\\
Consider the following, self-explanatory database schema about employees:

\begin{itemize}
\item
\text{employee(\textbf{employ\_name},street,city)}
\item
\text{works(\textbf{employ\_name},company\_name,salary)}
\item
\text{company(\textbf{company\_name},city)}
\item
\text{revenue(\textbf{company\_name},revenue)}
\item
\text{manages(\textbf{employ\_name},manager\_name)}
\end{itemize}
Primary key attributes are underlined. Foreign key attributes have the same names as the primary 
keys they reference. Express the folloing queries in SQL:
\begin{itemize}
\item Find the lowest salary of all maximum salaries payed in each company.
\item Companies may be located in serveral cities. Find the names of all companies located in every
 city in which Small Bank Corporation is located.
\item Find the names of the companies that have the most employees.
\item Give all managers of Firt Bank Corporation 1 10-percent raise unless the salary becomes greater than
 \$100,000; in such cases, give only 3-percent raise.
\item Find the names of all employees who earn more than the average salary of all employees of their company.
\item Find the names of companies that have the smallest payroll.
\item Find the company which has the maximum revenue per-employee(revene/totalNumberofEmployee)

\end{itemize}

\end{document}
